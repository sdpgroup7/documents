\documentclass[10pt, a4paper]{article} %Defines the document, we are forced to use 10pt

\usepackage{longtable}

\usepackage{fullpage} %makes the margins more reasonable for a 1 page report
\usepackage[a4paper,landscape]{geometry}

\begin{document}

\textbf{SDP Group 7 Team Performance review 2 - \today}


\begin{longtable}{c c c p{16cm}}
    \textbf{Name} &
    \textbf{Matric No} &
    \textbf{Score} &
    \textbf{Comments}\\
    \hline
    \endfirsthead


    \textbf{Name} &
    \textbf{Matric No} &
    \textbf{Score} &
    \textbf{Comments}\\
    \hline
    \endhead

    Wiktor Brodlo &
    s0927919 &
    6&
    I have developed a simulator of the robot to help the group with developing
    strategy. The simulator is almost done, it only requires some "glue" with
    the rest of the system.\\

    David Fraser &
    s0912336 &
    6&
    Worked on the high level planning for the AI and I am currently working on
    converting the plan from AI into robot commands.\\

    Radoslav Gabrovski &
    s0951580 &
    6& 
    I have been heavily involved in implementing and experimenting with four
    different methods to find the orientation of the robot, which produced
    unsatisfactory results. \\

    James Hulme &
    s0901522 &
    6& 
    For this milestone I attempted to improve the accuracy of the robot
    detection and to enable the vision system to detect orientation. I was
    partially sucessful in this: achieving an increase in robot detection.\\

    Dale Myers &
    s0942590 &
    9&
    This week my main contribution was to implement the testing system for the
    vision system along with trying out various methods of orientation
    detection of the robots. The testing system was rather successful but the
    orientation detection methods all failed to be better than our current
    method based on metrics from the testing system I developed.\\

    Laurie Picken &
    s0903587 &
    6&
    Provided an adaptable implementation of the A* path finding algorithm, to
    be used when dynamically creating plans. Continued to contribute to
    construction, providing quick fixes and analysis of the performance of
    different aspects of the robot.\\

    Darie Picu &
    s0935756 &
    6&
I built a spinner at the front of the robot for better control of the ball,
constructed a better braced frame and wrote a ball prediction class that takes
5 previous positions of the ball, computes the initial velocity and
deceleration and thus predicts the future X and Y coordinates of the ball. 
    \\

    Tomas Tauber &
    s0943263 &
    9&
    I wrote simple methods for getting the robots orientation without the exact
    Vision feedback and for navigating the robot as a backup solution, write
    unit tests for some helper methods I reused from the planning code.\\

    Christopher Williams &
    s0955088 &
    9&
    I have been working on the "higher level goal system" for the project:
    Planning and Strategy.\\

    Arran Wylde &
       s0811099 &
                0&
    Has not been seen or heard from. Ever.\\

\end{longtable}
\textbf{Good things} - A simulator is now in the works and other testing
frameworks are getting developed allowing us to test more easily. Strategy is
progressing well in preparation for the first friendly. Construction work has
been good with the robot not having to be rebuilt at all.

\textbf{Things to improve} - Attendance at meetings can be bad. Steps have been
taken to remind people of meetings. Detection of the robot still needs to be
completed.

\textbf{Goals for next Milestone} - A CI system like Jenkins to be working and
strategy being joined on to the main system so that it is dictating what the
robot is doing rather than being hard coded. More testing frameworks in place
and a working simulator to test the AI with.
\end{document}
