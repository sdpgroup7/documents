\section{Strategy}

%--------------------------------------------------------------CHRIS
%UML (yes this is going in appendix) - Chris
%we had more plans about how this was going to be done - Chris

Preliminary designs for Strategy were developed in meetings, some initial 
problems were:\newline
{\bf a)} How to represent the environment, {\bf b)} How to create some 
understanding of where the robot is with relation to 'a Plan', 
{\bf c)} How to build a test framework, {\bf d)} How to communicate a 
Strategy Model to team members and {\bf e)} What do we want a 'Plan' to be
\newline
The World state updated by the Vision system gave us an object orientated view
of the environment. This gave us a representation of the world we could use
to develop code and structure a strategy system around.

How to create some understanding of where the robot is in relation to a Plan
was a particularly difficult problem owing to the fact that there is a delay 
between where the robot is in reality and what is in the world state, and also 
because of the complication of needing to translate what a position means in the 
Strategy system into what a position means in the real world. This led to the 
conclusion that the Strategy subsystem should be kept simple, and we divided it 
up into further units which will be discussed subsequently: {\bf see Planning}.

The test frame work was initially going to be testing the logic and planning 
on a simulator, another design aspect seen frequently in previous year designs.
Also with Unit testing for individual code sections.

Communicating the design layout to the team was done in meetings, there were 
also UML diagrams created and placed on our group-public repository Github.
%APPENDIX REFERENCE TO SOME UML DESIGN

Lastly developing what a 'Plan' should be went through several cycles, initially
the plan was to be driven by the state of the environment, where we would have 
relationships between all the major objects and some event based reaction. 
However looking at a previous groups code this meant creating huge amounts of
conditions, which we believed to be particularly difficult and an overly 
complicated design. Finally we settled on a 'Ball driven' design where by the 
path finding and decision making would be made primarily on where the ball is
in relation the to oppositions goal line.
%--------------------------------------------------------------CHRIS

\subsection{Planning}


%--------------------------------------------------------------CHRIS
%some introduction to planning...

We solved the problem of where the robot is in relation to the Plan quite simply
by ignoring it. That is to say we developed a model that would create a Plan
as fast as possible, as often as possible, on what information was in the World
State. This meant we were always designing a Plan from the latest known position 
of the robot.

Planning was driven by an Object called Target Decision, which passes a target
point to the A* which in turn creates a path:

%Target Decision (Chris):

Target decision is the decision structure for the 'Plan', based on the ball 
position, it tries finds a point that the robot can get to, based on certain 
restrictions; the oppositions position and the pitch walls. Failing a obvious
shot into the goal, it will create the option to ricochet the ball off a pitch
wall. If however the opposition is in a threatening position, Target Decision
will drive the A* to create a path back to our own goal line.

%Nav point - Subsumption Arch - Interception

One obvious part of a Plan is the path required...
%over to you Laurie?
%This doesn't make any sense, one person should write the A* part
%discuss why we took A* over potential fields, that we wanted an A* from grp 3
%that our initial A* had a limitation with the binary objects problem, and that
%we updated it with cost-objects instead
%--------------------------------------------------------------CHRIS

Without a simulator it was difficult to see the flow of information within our program. We had immense troubles in trying to see what the strategy system was actually “thinking”. The solution was rather simple. An overlay was superimposed on top of the vision feed which showed how the strategy system viewed the world. We could effectively see the locations of the robots, the boundaries around them, the paths that were generated by A* and the nav-point described above. 
%SHOW IMAGE FOR THIS. 


\subsection{Control Interface}

With the method of planning being set as A-star we needed some method to
convert the plan given by the strategy systems into commands the robot could
perform. Our objective for this subsystem was to find a simple algorithm that
would create a smooth robotic motion which could respond without disruption to
changes in strategy's plan.

Point to point navigation, the robot turning and moving to each point on the
A-star plan, was dismissed early on. It would not provide us with the smooth
motion we wanted as it would have stopped at each point on the path whenever it
needed to make a significant turn.

We reasoned that the best way to solve the problem of the stop start motion of
the point to point navigation was to apply smoothing to the path given by
A star. A variation of A star called Theta star provided this by selecting
a lookahead distance which allowed some points on the path to be missed out
thus creating a smoother path.

The Pure Pursuit algorithm was found to satisfy our needs. It is used
frequently in autonomous vehicles with non holomonic
movement for path tracking\cite{agvpp}\cite{coulterpp}. In our case we are
using it to track the path given to us by A star.

As the robot is constantly moving in some sort of arc under pure pursuit this
eliminates the problems caused by point to point navigation. The motion is
smooth and uninterrupted. Pure pursuit also only considers a limited distance
in front of it the plan can be changed as the situation on the pitch changed.

%--------------------------------------------------------------CHRIS
\subsection{Discussion - Planning}

%what we would do differently and why - Chris
%why did we not use group 3 code, or any other group’s code - Chris

Two major costs in regards to effort put into Planning were the lack of a 
simulator, and our work on A*. 

The simulator was the biggest cost to this project, owing to the fact we had
no way of testing out logic and maths in a controllable environment. We should
have taken a previous years simulator and adapted it to our requirements as
early as possible. 

Our final A* design still has a flaw, using cost based objects required the 
g-cost of the A* to be reduced else the A* would function incorrectly, thus a 
parameter needed to be used to reduce the weighting of g-cost in the A*, which 
meant the given path was not always optimal. Some function that optimises the A* 
path after creation would be necessary to fix this flaw, one that has latterly 
been discovered in a previous years code (Group 3). Discovered too late 
in the project, we should have instead opted to use a previous years A* from the 
start and spent our time optimising this code instead of writing our own 
(again Group 3).

Target Decision was the only portion of this code that could not have been 
taken from previous years, as there was no suitable code (suitable defined by:
well commented, written in Java, and not simplistic if/else statements). This 
area should be given far more consideration, our subsumption architecture is
simplistic, but it has scope for more intelligent use, such as: The ability to
change from a Penalty Mode into a Free play mode and the reverse of this 
scenario, the ability to consider the pitch in sections, and thus create a Plan
for when the ball is in the corner (a weakness for most oppositions) and lastly 
to utilize our spinners a mode where we can dribble the ball by using the 
'Target' and 'NavPoint' intelligently.
%--------------------------------------------------------------CHRIS


