\section{Strategy}

%--------------------------------------------------------------CHRIS
%UML (yes this is going in appendix) - Chris
%we had more plans about how this was going to be done - Chris

Preliminary designs for Strategy were developed in meetings, some initial 
problems were:
how to create some understanding of where the robot is in relation to a Plan. 
This
was a particularly difficult problem owing to the fact that there is a delay 
between where the robot is in reality and what is in \texttt{WorldState}, also 
because of the complication of needing to translate what a position means in the 
Strategy system into what a position means in the real world. This led to the 
conclusion that the Strategy subsystem should be kept simple, so we divided it 
up into further units which will be discussed subsequently in
section~\ref{sec:planning}.

Communicating the design layout to the team was done in meetings, there were 
also UML diagrams created and shared.
%APPENDIX REFERENCE TO SOME UML DESIGN
%TODO

\subsection{Planning}
\label{sec:planning}

%some introduction to planning...

We solved the problem of where the robot is in relation to the Plan by ignoring
it. We developed a model that would create a Plan as fast as possible, based on
what information was in the \texttt{WorldState}. This meant we were always
designing a Plan from the latest known position of the robot.

Planning was driven by an object called Target Decision, which passes a target
point to the A* which creates a path.

%Target Decision (Chris):

Target Decision decides which point to drive the robot to, based on the ball
position. It uses certain restrictions such as the opposition's position and
the pitch walls to calculate the position it would be most useful for the robot
to be in. Failing an obvious shot into the goal it will create the option to
ricochet the ball off a pitch wall. If however the opposition is in a
threatening position Target Decision will direct the A* to create a path back to
our own goal line.

%Nav point - Subsumption Arch - Interception

%-Added by Laurie
We used the A* algorithm for path finding.  The reason we chose A* was the
positive feedback from groups who had used it last year, in particular group
three who said it was the most effective part of their planning sub-system.
Our first version of A* had a flaw in that the obstacles used for avoidance
were binary; a node was either a reachable position or it wasn't. 
If the ball happened to be inside an obstacle then we could not create a path
to the ball, path-finding would fail. This meant our A* had to be re-written to
give obstacles a cost value. In this way the algorithm will try to
avoid obstacles, but can still plot a path to a ball inside an obstacle.
%-Image of the overlay doing this would be good here.

\subsection{Control Interface}

To realise the A* path into robot movement we used Pure Pursuit;
our objective for this sub-system was to find an algorithm that
would create a smooth motion which could respond without stopping to
new changes in the plan.

Point to point navigation; the robot turning and moving to each point on the A*
path, was dismissed early on as it could not provide us with smooth motion.
Literal A* path translation would have stopped at each point on the path
whenever it needed to make a turn.

We reasoned that the best way to solve the problem of the stop start motion of
the point to point navigation was to apply smoothing to the path. 

The Pure Pursuit is used frequently in autonomous vehicles with non holonomic
movement for path acquisition tracking\cite{agvpp}\cite{coulterpp}. In our case
we are using it to track the path given to us by A*.

Under pure pursuit the robot is constantly moving in an arc. This
eliminates the problems caused by point to point navigation. The motion is
smooth and uninterrupted. As pure pursuit also only considers a limited distance
in front of it the robot need not stop as the plan changes. The arcing movement
is also sympathetic to the operation of the spinners.

%--------------------------------------------------------------CHRIS
\subsection{Discussion - Planning}

%what we would do differently and why - Chris
%why did we not use group 3 code, or any other group’s code - Chris

Two major costs in developing and realising Strategy were the lack of a 
simulator and our work on A*. 

The simulator was the biggest cost to this project, owing to the fact we had
no way of testing in a controllable environment. We should
have taken a previous year's simulator and adapted it to our requirements as
early as possible. 

However as a solution to not having a simulator, an overlay was superimposed on
top of the vision feed which showed the plans created by the strategy system in
real time. We could see the locations of the robots, the ball, and the paths
that were generated by A*. 

Our final A* design still has a flaw as seen on the overlay: using cost based
objects required the changes to the standard implementation of A* meaning the
given path was sub-optimal. Some function that optimises the A* path after
creation would be necessary to fix this flaw, one that has latterly been
discovered in a previous years code (Group three). We should have instead opted
to use a previous years A* and planning code from the start and spent our time
optimising this code instead of writing our own.

Target Decision was the only portion of this code that could not have been
taken from previous years, as there was no suitable code\footnote{suitable
defined by: well commented, written in Java, and not simplistic if/else
statements}. This area, higher level decision making, should be given more time
and consideration.  Our subsumption architecture is simplistic, but with scope
for more features: The ability to change from a Penalty Mode into a Free play
mode, the ability to create a Plan for when the ball is in a corner\footnote{a
weakness for most opposition teams}, and lastly the ability to take advantage
our spinners with a mode where we can dribble the ball.
%--------------------------------------------------------------CHRIS


