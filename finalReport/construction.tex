\section{Construction}

Looking at videos from previous years of SDP, we were sure that we wanted our robot to have a very strong chassis in order to sustain possible collisions and as big as possible to make defending penalties easier. After looking at robots from previous years on the course webpage and in Garry's office, a basic three wheeled design was born. It had two wheels at the sides and a metal rotating ball at the back for stability.

%-Added by Laurie
By positioning the drive wheels towards the front of the robot, we achieve a higher level of control when making precise turns to change the direction of the kicker. Our initial design used only one wheel on either side, but we found using two increased traction and thus made fast accelerations and decelerations smoother.

Throughout the assembling process, as a result from tests of moving the robot on the pitch, it became clear that a low and middle-positioned centre of mass was required for a stable robot. This led to the following construction with a very strong and reinforced chassis (INSERT A REFERENCE TO  IMAG0217-001.jpg). It proved its worth, when an unexpected stress test was enforced on the robot, as a team member drove it off one of the desks in the lab. There was no severe damage the construction. 

%-Added by Laurie - sorry Darie I've just seen you said you're going to do this, feel free to delete it
Given the difficulties in keeping control of the ball when dribbling, we built a pair of spinning wheels into the front of our robot. They turn in the same direction as the drive wheels, imparting backspin on the ball and preventing it from rolling away. We opted to drive the spinners via gearing from our drive motors, as opposed to using a separate motor. This meant we avoided having to use a second brick or additional motor (with multiplexer) in the design, which would take up a lot of space and add unnecessary weight to the construction. Two small prongs are used either side of the spinners to keep the ball in contact with them and prevent it rolling off to the side. We also chose softer hollow wheels instead of solid ones, as they absorb much of the initial impact with the ball. The result is a very effective system for dribbling, we can intercept the ball and keep control of it at high speeds and perform sharp turns whilst in possession. (Reference to IMAG0166-001.jpg)

%-Added by Laurie
In order to achieve the greatest swing and power in kicking, the kicker was positioned directly on top of the extended arm. Again, a lot of experimenting with different designs led to this conclusion. We opted for the highest gear ratio possible (using two cogs), to transfer the maximum possible angular velocity to the bottom of the kicker. This meant a very lightweight design for the kicker was optimal, to reduce the force needed to accelerate it to full speed. Constructing a powerful kicker to work around the spinners was the hardest challenge we had to overcome in the construction team. After finally putting together a functional kicker, we realised it was not particularly strong because the period of time during a kick in which the ball was in contact with the kicker, the motor had not yet reached its full speed. We fixed this by setting back the actual part of the kicker that would make contact with the ball, so it sits just in front of the spinners. This meant when we executed a kick the kicker had already accelerated to its maximum speed when making contact with the ball, producing a very fast kick that could be executed almost immediately.

% Rado, I think this next bit would be more suited in the paragraph about the chassis. Or is it specifically to do with the kicker?
% Yes, I moved it to the part with the chassis - Rado 

%-What would we do differently, are we happy with it?
In regard to what we could have done differently, we could have documented the initial design processes much better. However, after our first performance review, extra care was taken in documenting our work (May be link to some pics in appendix?). Overall, our robot had some unique features and are very happy its design.
