\section{Construction}

Looking at videos from previous years of SDP, we were sure that we wanted our robot to be with a very strong chassis, in order to sustain possible impacts, and as big as possible, to make defending penalties easier. Looking at robots from previous years on the course webpage and in Garry’s office, a basic three wheeled design was born, that had two wheels at the sides and a metal rotating ball at the back for stability. Throughout the assembling process and as a result from tests of moving the robot on the pitch, it became clear to me that a low and middle positioned centre of mass was required for a stable robot. This lead to the following construction with a very strong and reinforced chassis(pic): 

I also insisted that the kicker is positioned directly on top the extended arm that kicks in order to achieve maximum swing and power to the kick. Again a lot of experimenting with different designs led to this conclusion. It also proved its worth, when an unexpected stress test was enforced on the robot, as a team member drove it off one of the desks in the lab. There was no severe damage the construction.(Laurie feel free to edit this) 
