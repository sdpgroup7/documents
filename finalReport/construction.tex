\section{Construction}

Looking at videos from previous years of SDP, we were sure that we wanted our robot to be with a very strong chassis in order to sustain possible collisions and as big as possible to make defending penalties easier. After looking at robots from previous years on the course webpage and in Garry's office, a basic three wheeled design was born. It had two wheels at the sides and a metal rotating ball at the back for stability. Throughout the assembling process, as a result from tests of moving the robot on the pitch, it was becoming clear that a low and middle-positioned centre of mass was required for a stable robot. This led to the following construction with a very strong and reinforced chassis (INSERT A REFERENCE TO A PICTURE). 

In order to achieve the greatest swing and power in kicking, the kicker was positioned directly on top of the extended arm. Again, a lot of experimenting with different designs led to this conclusion. It also proved its worth, when an unexpected stress test was enforced on the robot, as a team member drove it off one of the desks in the lab. There was no severe damage the construction.(Laurie feel free to edit this) 
