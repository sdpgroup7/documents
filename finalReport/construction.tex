\section{Construction}

Following research from previous years we decided upon a large chassis for;
strength, and size when defending against penalties. 

%-Added by Laurie
We designed a three wheeled robot with two large drive wheels at the front with
a ball bearing at the rear. Tested on the pitch this proved to be manoeuvrable
and stable. 

To increase ball control we built a pair of  wheels onto the front of
our robot which turn in the same direction as the drive wheels. This creates
'backspin' on the ball and prevents the ball from rolling away on contact.  We
opted to drive the spinners via gearing from our drive motors, as opposed to
using a separate motor.  We chose soft wheels over hard wheels, as they
absorb the initial impact as shown in figure~\ref{fig:spinners} in
appendix~\ref{apx:robot}.

%-Added by Laurie
Constructing a powerful kicker to work around the spinners was the
hardest challenge as the spinners occupied the space where the kicker was
previously.

Overall, our robot had a unique feature,
the spinners, and we are happy its design. If doing it again, a better
approach would be duplicating a robot from last year.
