\section{Group Dynamics}

\subsection{Team Meetings}

At the beginning of the project the team agreed to meet once a week. As the
project progressed\footnote{Around the second milestone} it became clear that
the amount of work getting done and the need for discussions and decisions that
could not be done within the confines of an email conversion.

\subsection{Issue resolution}

When there was an issue which could not be resolved within a team meeting or
there was significant disagreement on a subject a sub team would be created.
The sub team would consist of three or four people from various
teams\footnote{Control, Vision, Strategy, Simulator, and Construction} who the
outcome of the decision would have an effect on.

For example, when it was realised that there was problems with control code,
people representing control, control interface, and simulator were present. It
would not have been useful for the entire team to discuss the subject. The
people represented both had an understanding of the problems with the current
control system and their work would be directly affected by the changes that
would come out of that meeting.

The decisions taken by the meeting would be relayed to the rest of the team and
anything that needed done assigned to the appropriate parties.

\subsection{Simulator}

The largest failing with regard to teamwork and team communication was the
assignment of only one person initially to the development of the simulator.
The team member was not present for some of the subsequent team meetings.
Progress reports from email suggested that development was going well The
simulator was written in a language than very few members of the team knew (C).
It was then near impossible for another member of the team to continue When the
simulator was demonstrated it was clear that it was not usable and was not
fulfilling its purpose.

More man power was directed to the simulator when the problem was realised
however it was too late to have it salvaged ready for useful testing and it was
abandoned.

\subsection{Team Movement}

The team as whole naturally broke down into sub-teams for each of the
sub-systems. As parts were completed and as man hours were required on other
parts of the project were moved off and were trained in another subsystem. As
a result of this there were one or two people who were well versed in the
system as whole through their work on different sub-systems.

\subsection{Conclusions}

We learned that team communication is the most important element of a project
of this scale. We feel that on the whole we resolved problems quickly and
effectively. We met regularly as a team and those meetings were fruitful.

In terms of what we as a team would have done differently:
\begin{itemize}
    \item Assign more than one person initially to the simulator development
    \item At the start of the project appoint a technical lead developer
    different from the team manager. One person overseeing the technical
    direction of the project ensuring that integration happened properly.
\end{itemize}

\end{itemize}

