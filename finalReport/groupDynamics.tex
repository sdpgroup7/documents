\section{Group Dynamics}

\subsection{Team Meetings}

At the beginning of the project the team agreed to meet once a week. The
meetings consisted if a SCRUM like discussion of current issues. As the
project progressed\footnote{Around the second milestone} it became clear that
team needed to spend more time together. As a result the number of
meetings was increased to biweekly and agreement that more time would be spend
in labs on Wednesdays, dedicating our time only to SDP.

\subsection{Issue resolution}

When there was an issue which could not be resolved within a team meeting or
there was significant disagreement on a subject a sub team would be created.
The sub team would consist of three significant people from various
teams.

For example, when it was realised that there were problems with the control
code, people representing control, control interface, and simulator were
present. It would not have been useful for the entire team to discuss the
subject. The people represented in this sub team both had an understanding of
the problems with the current control system and their work would be directly
affected by the changes that would come out of that meeting.

The decisions taken by the meeting would be relayed to the rest of the team and
anything that needed done assigned to the appropriate parties.

\subsection{Simulator}

The largest failing with regard to teamwork and team communication was the
assignment of only one person initially to the development of the simulator.
The team member was not present for some of the subsequent team meetings.
Progress reports from email suggested that development was going well The
simulator was written in a language than very few members of the team knew (C).
It was then near impossible for another member of the team to continue When the
simulator was demonstrated it was clear that it was not usable and was not
fulfilling its purpose.

More man power was directed to the simulator when the problem was realised
however it was too late to have it salvaged ready for useful testing and it was
abandoned.

\subsection{Conclusions}

We learned that team communication is the most important element of a project
of this scale. We feel that on the whole we resolved problems quickly and
effectively. We met regularly as a team and those meetings were fruitful.

In terms of what could have done differently: assign more people to the
simulator development and appoint a technical lead developer, who is not the
team manager, overseeing the technical direction of the project ensuring that
integration happened properly.


