\section{Control}
In the last year's code repository, several teams reported their Control/Communication sub-system had occasional problems. Group 5, however, did not mention any such problems, their Control code was easy to follow (clearly defined communication interface, integer operation codes for controlling the robot and so on) and used leJOS, an actively developed firmware and framework for LEGO Mindstorms in Java. For these reasons, we based our Control sub-system on their code. After upgrading to the latest leJOS version 0.9.1 from 0.85 (it promised bug fixes, less memory usage, reduced power consumption when idle, better Java language support, more of useful tools and classes\cite{lejos}), some parts of the reused code had to be rewritten, because they used deprecated constructs. Using the class \textit{OdometryPoseProvider}, a simple fallback navigation was made. It saved the robot's initial position, in front of our goal, and would move the robot directly there in the event of a lost bluetooth connection.

\subsection{What would we do differently?}
All teams that used leJOS last year and most likely even this year designed their NXT brick code using one of two approaches or their combination: competing or collaborating threads and event-triggering. Even though it is valid and working in most cases, it often results in a ``spaghetti code'' and is not recommended by leJOS, as we later discovered. The recommended design pattern for autonomous robots is the \textbf{behaviour control model}\cite{behaviour}. In this model, each robot's task is defined in its own class that implements the \textsl{Behavior} (the API's spelling) interface and it is clear what each behaviour does and when, and how it can be suppressed. These behaviours are then passed with fixed priorities to the \textsl{Arbitrator} class that regulates which behaviour should be active at any time. If more behaviours want to take control, only the one with the highest priority will become active and the current one will be suppressed.

There are many advantages of this pattern: behaviours can be easily added, removed, or separated for testing; anyone familiar with this model can understand the code (i.e. more likely to get help from the leJOS community); it is guaranteed that only one behaviour can be in control at any time; behaviours do not make any assumptions about their priorities, so that priorities can be changed without reprogramming behaviours. Furthermore, it would then be easier to integrate the whole system with leJOS framework tools for robotics, including, for example, the A* search algorithm that we (and many teams) implemented on our own. Using a 3$^{\textrm{rd}}$ party API in this scenario would enforce the same \textbf{consistent world state representation} and prevent the project from some problems we had due to the fact that people used different angle systems and units of measure in their algorithms.