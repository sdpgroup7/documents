\section{Introduction}

Our goal for this project was to create a robot that could win the final
tournament. The entire team met and decided on a common programming language,
Java, a version control system, git, a team leader, and what sub-systems people
wanted to work on\footnote{Full details in appendix~\ref{apx:roles}}.

We expected the project to be hard technically. As it turned out the things we
expected to be trivial, communicating with each other, were the hardest and
many of the technical things we expected to find hard were less so than we
expected.
%--------------------------------------------------------------CHRIS
%introduction to the flow diagram of subsystems and why (maybe) it breaks down 
%this way. - Chris
%!!!!!!!!!!!!!!!!!!
%Can someone do me a decent flow diagram, as our current one is shit!

Research into previous year's code revealed a common structural layout CITE THE FIGURE. 
Nearly all previous year groups broke their project into obvious subsections; 
A shared object which contained environmental data (A WorldState), a vision 
system that fed and updated the shared object, a strategy system that was 
considered the 'artificial intelligence' (although most were a variant on subsumption architectures) taking its cues from the shared object, and a control 
system that interfaces between the computer system and the robot: between the 
strategy and the robot. These independent subsystems in some cases can be broken 
down further and will be discussed subsequently. 
%--------------------------------------------------------------CHRIS
