\documentclass{article}

\usepackage{pdflscape}
\usepackage{longtable}

\begin{document}

\title{Risk Assessment}
\author{SDP Group 7\\
The answer to life, the universe and every over six
}
\maketitle

\section{Introduction}

This document identifies risks, both internal and external, to the project. It
also sets out measures which we have implemented to prevent the risk
happening and steps if the event does indeed happen.

Risks are rated from 1 to 5 with 1 being the least severe (minor) and 5 being
the most severe (catastrophic).

\section{Assessment}
%
%\begin{landscape}
%
%    \begin{tabular}{ p{4cm}  p{2cm} p{2cm} p{4cm} p{4cm} }
%        Risk & Rating & Likelyhood & Measures to prevent & Steps in event of\\
%        \hline
%
%        Communication failure over bluetooth to robot &
%        3 &
%        20\% &
%        The robot
%        has a hardcoded fallback mode so if the communication link is lost it
%        will use odometry to drive back to its starting point and act as
%        a goalie. &
%        Allow the hard coded method to be used and try and establish the
%        communication link again.\\
%        \\
%        Some exception on the brick will cause the control system on the brick
%        to stop. &
%        4 &
%        30\% &
%        As many as possible exceptions on the brick will be caught and handled.
%        &
%        Our robot will probably stop and restart. It is very probable we will
%        lose the match.\\
%        \\
%        The battery runs out during a match or milestone. &
%        4 &
%        5\% &
%        Before each match and milestone the LithIon battery will be charged
%        overnight. The AA battery's will be charged as well to be used as
%        a reserve and can be swapped out if the battery level runs too low.&
%        %
%        There will be reduced power output from the motors as the batteries
%        approached the end of their power output. The robot will be dead and we
%        will maybe be able to complete the milestone our match will be null and
%        void.\\
%        \\
%        Github failure &
%        5 &
%        1\% &
%        Since git is a distributed version control system, that is each person
%        has a full copy of the repository plus the full commit history. We
%        would be able to recover a close version of the code close to the point
%        of loss. &
%        %
%        The branches of all the code will all be slightly different. A new
%        origin server would be set up and the various teams and their branches
%        push to it. The previous milestone will have been preserved on
%        everybody's local repository so in the worst case we can fall back to
%        that.\\
%        \\
%        Robot falls off a table during testing &
%        3&
%        10\%
%        The robots motion will never be tested on a tabletop, always on the
%        floor or on one of the pitches. On the floor there is no risk of
%        falling and on the pitches there are barriers around the sides to stop
%        this happening. Every build of the robot will be photographed so that
%        we can rebuild from that if necessary.
%        %
%        The robot would obviously have to be rebuilt and would take at most two
%        hours to do this. If this happened near a milestone or match this would
%        be far more serious.\\
%        \\
%
%
%        
%
%
%
%    \end{tabular}

\begin{longtable}{p{2cm} p{2cm} p{4cm} p{4cm}}

    \caption{Risk Analysis \label{table:riskAnalysis}}\\
    
    %This is the header for the first page of the table
    
    {\textbf{Risk}}&
    {\textbf{Rating}}&
    {\textbf{Measures to prevent}}&
    {\textbf{Steps in event of}}\\
    \hline
    \endfirsthead

    % Top of every page after that

    \multicolumn{3}{c}{{\tablename} \thetable{} -- Continued} \\[0.5ex]
    {\textbf{Risk}}&
    {\textbf{Rating}}&
    {\textbf{Measures to prevent}}&
    {\textbf{Steps in event of}}\\
    \hline
    \endhead

    %This is the footer for all pages except the last page of the table...

    \multicolumn{3}{l}{{Continued on Next Page\ldots}} \\
    \endfoot

    %Footer on the last page of the table

    \hline
    \endlastfoot

    %Data
    
\end{longtable}





\end{landscape}
\end{document}

