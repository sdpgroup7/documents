\documentclass{article}

\usepackage{pdflscape}

\begin{document}

\title{Risk Assessment}
\author{SDP Group 7\\
The answer to life, the universe and every over six
}
\maketitle

\section{Introduction}

This document identifies risks, both internal and external, to the project. It
also sets out measures which we have implemented to prevent the risk
happening and steps if the event does indeed happen.

Risks are rated from 1 to 5 with 1 being the least severe (minor) and 5 being
the most severe (catastrophic).

\section{Assessment}

\begin{landscape}

    \begin{tabular}{ p{4cm}  p{2cm} p{2cm} p{4cm} p{4cm} }
        Risk & Rating & Likelyhood & Measures to prevent & Steps in event of\\
        \hline

        Communication failure over bluetooth to robot &
        3 &
        20\% &
        The robot
        has a hardcoded fallback mode so if the communication link is lost it
        will use odometry to drive back to its starting point and act as
        a goalie. &
        Allow the hard coded method to be used and try and establish the
        communication link again.\\
        \\
        Some exception on the brick will cause the control system on the brick
        to stop. &
        4 &
        30\% &
        As many as possible exceptions on the brick will be caught and handled.
        &
        Our robot will probably stop and restart. It is very probable we will
        lose the match.\\
        \\
        The battery runs out during a match or milestone. &
        4 &
        5\% &
        Before each match and milestone the LithIon battery will be charged
        overnight. The AA battery's will be charged as well to be used as
        a reserve and can be swapped out if the battery level runs too low.&
        %
        There will be reduced power output from the motors as the batteries
        approached the end of their power output. The robot will be dead and we
        will maybe be able to complete the milestone our match will be null and
        void.\\
        \\
        Github failure &
        5 &
        1\% &
        Since git is a distributed version control system, that is each person
        has a full copy of the repository plus the full commit history. We
        would be able to recover a close version of the code close to the point
        of loss. &
        %
        The branches of all the code will all be slightly different. A new
        origin server would be set up and the various teams and their branches
        push to it. The previous milestone will have been preserved on
        everybody's local repository so in the worst case we can fall back to
        that.\\
        \\
        Robot falls off a table during testing &
        3&
        10\%

        



    \end{tabular}
\end{landscape}
\end{document}

